\documentclass{ximera}

%\usepackage{todonotes}

\newcommand{\todo}{}

\usepackage{esint} % for \oiint
\ifxake%%https://math.meta.stackexchange.com/questions/9973/how-do-you-render-a-closed-surface-double-integral
\renewcommand{\oiint}{{\large\bigcirc}\kern-1.56em\iint}
\fi


\graphicspath{
  {./}
  {ximeraTutorial/}
  {basicPhilosophy/}
  {functionsOfSeveralVariables/}
  {normalVectors/}
  {lagrangeMultipliers/}
  {vectorFields/}
  {greensTheorem/}
  {shapeOfThingsToCome/}
  {dotProducts/}
  {partialDerivativesAndTheGradientVector/}
  {../productAndQuotientRules/exercises/}
  {../normalVectors/exercisesParametricPlots/}
  {../continuityOfFunctionsOfSeveralVariables/exercises/}
  {../partialDerivativesAndTheGradientVector/exercises/}
  {../directionalDerivativeAndChainRule/exercises/}
  {../commonCoordinates/exercisesCylindricalCoordinates/}
  {../commonCoordinates/exercisesSphericalCoordinates/}
  {../greensTheorem/exercisesCurlAndLineIntegrals/}
  {../greensTheorem/exercisesDivergenceAndLineIntegrals/}
  {../shapeOfThingsToCome/exercisesDivergenceTheorem/}
  {../greensTheorem/}
  {../shapeOfThingsToCome/}
  {../separableDifferentialEquations/exercises/}
}

\newcommand{\mooculus}{\textsf{\textbf{MOOC}\textnormal{\textsf{ULUS}}}}

\usepackage{tkz-euclide}\usepackage{tikz}
\usepackage{tikz-cd}
\usetikzlibrary{arrows}
\tikzset{>=stealth,commutative diagrams/.cd,
  arrow style=tikz,diagrams={>=stealth}} %% cool arrow head
\tikzset{shorten <>/.style={ shorten >=#1, shorten <=#1 } } %% allows shorter vectors

\usetikzlibrary{backgrounds} %% for boxes around graphs
\usetikzlibrary{shapes,positioning}  %% Clouds and stars
\usetikzlibrary{matrix} %% for matrix
\usepgfplotslibrary{polar} %% for polar plots
%\usepgfplotslibrary{fillbetween} %% to shade area between curves in TikZ
\usetkzobj{all}
\usepackage[makeroom]{cancel} %% for strike outs
%\usepackage{mathtools} %% for pretty underbrace % Breaks Ximera
%\usepackage{multicol}
\usepackage{pgffor} %% required for integral for loops



%% http://tex.stackexchange.com/questions/66490/drawing-a-tikz-arc-specifying-the-center
%% Draws beach ball
\tikzset{pics/carc/.style args={#1:#2:#3}{code={\draw[pic actions] (#1:#3) arc(#1:#2:#3);}}}



\usepackage{array}
\setlength{\extrarowheight}{+.1cm}
\newdimen\digitwidth
\settowidth\digitwidth{9}
\def\divrule#1#2{
\noalign{\moveright#1\digitwidth
\vbox{\hrule width#2\digitwidth}}}





\newcommand{\RR}{\mathbb R}
\newcommand{\R}{\mathbb R}
\newcommand{\N}{\mathbb N}
\newcommand{\Z}{\mathbb Z}

\newcommand{\sagemath}{\textsf{SageMath}}


%\renewcommand{\d}{\,d\!}
\renewcommand{\d}{\mathop{}\!d}
\newcommand{\dd}[2][]{\frac{\d #1}{\d #2}}
\newcommand{\pp}[2][]{\frac{\partial #1}{\partial #2}}
\renewcommand{\l}{\ell}
\newcommand{\ddx}{\frac{d}{\d x}}

\newcommand{\zeroOverZero}{\ensuremath{\boldsymbol{\tfrac{0}{0}}}}
\newcommand{\inftyOverInfty}{\ensuremath{\boldsymbol{\tfrac{\infty}{\infty}}}}
\newcommand{\zeroOverInfty}{\ensuremath{\boldsymbol{\tfrac{0}{\infty}}}}
\newcommand{\zeroTimesInfty}{\ensuremath{\small\boldsymbol{0\cdot \infty}}}
\newcommand{\inftyMinusInfty}{\ensuremath{\small\boldsymbol{\infty - \infty}}}
\newcommand{\oneToInfty}{\ensuremath{\boldsymbol{1^\infty}}}
\newcommand{\zeroToZero}{\ensuremath{\boldsymbol{0^0}}}
\newcommand{\inftyToZero}{\ensuremath{\boldsymbol{\infty^0}}}



\newcommand{\numOverZero}{\ensuremath{\boldsymbol{\tfrac{\#}{0}}}}
\newcommand{\dfn}{\textbf}
%\newcommand{\unit}{\,\mathrm}
\newcommand{\unit}{\mathop{}\!\mathrm}
\newcommand{\eval}[1]{\bigg[ #1 \bigg]}
\newcommand{\seq}[1]{\left( #1 \right)}
\renewcommand{\epsilon}{\varepsilon}
\renewcommand{\phi}{\varphi}


\renewcommand{\iff}{\Leftrightarrow}

\DeclareMathOperator{\arccot}{arccot}
\DeclareMathOperator{\arcsec}{arcsec}
\DeclareMathOperator{\arccsc}{arccsc}
\DeclareMathOperator{\si}{Si}
\DeclareMathOperator{\scal}{scal}
\DeclareMathOperator{\sign}{sign}


%% \newcommand{\tightoverset}[2]{% for arrow vec
%%   \mathop{#2}\limits^{\vbox to -.5ex{\kern-0.75ex\hbox{$#1$}\vss}}}
\newcommand{\arrowvec}[1]{{\overset{\rightharpoonup}{#1}}}
%\renewcommand{\vec}[1]{\arrowvec{\mathbf{#1}}}
\renewcommand{\vec}[1]{{\overset{\boldsymbol{\rightharpoonup}}{\mathbf{#1}}}}
\DeclareMathOperator{\proj}{\mathbf{proj}}
\newcommand{\veci}{{\boldsymbol{\hat{\imath}}}}
\newcommand{\vecj}{{\boldsymbol{\hat{\jmath}}}}
\newcommand{\veck}{{\boldsymbol{\hat{k}}}}
\newcommand{\vecl}{\vec{\boldsymbol{\l}}}
\newcommand{\uvec}[1]{\mathbf{\hat{#1}}}
\newcommand{\utan}{\mathbf{\hat{t}}}
\newcommand{\unormal}{\mathbf{\hat{n}}}
\newcommand{\ubinormal}{\mathbf{\hat{b}}}

\newcommand{\dotp}{\bullet}
\newcommand{\cross}{\boldsymbol\times}
\newcommand{\grad}{\boldsymbol\nabla}
\newcommand{\divergence}{\grad\dotp}
\newcommand{\curl}{\grad\cross}
%\DeclareMathOperator{\divergence}{divergence}
%\DeclareMathOperator{\curl}[1]{\grad\cross #1}
\newcommand{\lto}{\mathop{\longrightarrow\,}\limits}

\renewcommand{\bar}{\overline}

\colorlet{textColor}{black}
\colorlet{background}{white}
\colorlet{penColor}{blue!50!black} % Color of a curve in a plot
\colorlet{penColor2}{red!50!black}% Color of a curve in a plot
\colorlet{penColor3}{red!50!blue} % Color of a curve in a plot
\colorlet{penColor4}{green!50!black} % Color of a curve in a plot
\colorlet{penColor5}{orange!80!black} % Color of a curve in a plot
\colorlet{penColor6}{yellow!70!black} % Color of a curve in a plot
\colorlet{fill1}{penColor!20} % Color of fill in a plot
\colorlet{fill2}{penColor2!20} % Color of fill in a plot
\colorlet{fillp}{fill1} % Color of positive area
\colorlet{filln}{penColor2!20} % Color of negative area
\colorlet{fill3}{penColor3!20} % Fill
\colorlet{fill4}{penColor4!20} % Fill
\colorlet{fill5}{penColor5!20} % Fill
\colorlet{gridColor}{gray!50} % Color of grid in a plot

\newcommand{\surfaceColor}{violet}
\newcommand{\surfaceColorTwo}{redyellow}
\newcommand{\sliceColor}{greenyellow}




\pgfmathdeclarefunction{gauss}{2}{% gives gaussian
  \pgfmathparse{1/(#2*sqrt(2*pi))*exp(-((x-#1)^2)/(2*#2^2))}%
}


%%%%%%%%%%%%%
%% Vectors
%%%%%%%%%%%%%

%% Simple horiz vectors
\renewcommand{\vector}[1]{\left\langle #1\right\rangle}


%% %% Complex Horiz Vectors with angle brackets
%% \makeatletter
%% \renewcommand{\vector}[2][ , ]{\left\langle%
%%   \def\nextitem{\def\nextitem{#1}}%
%%   \@for \el:=#2\do{\nextitem\el}\right\rangle%
%% }
%% \makeatother

%% %% Vertical Vectors
%% \def\vector#1{\begin{bmatrix}\vecListA#1,,\end{bmatrix}}
%% \def\vecListA#1,{\if,#1,\else #1\cr \expandafter \vecListA \fi}

%%%%%%%%%%%%%
%% End of vectors
%%%%%%%%%%%%%

%\newcommand{\fullwidth}{}
%\newcommand{\normalwidth}{}



%% makes a snazzy t-chart for evaluating functions
%\newenvironment{tchart}{\rowcolors{2}{}{background!90!textColor}\array}{\endarray}

%%This is to help with formatting on future title pages.
\newenvironment{sectionOutcomes}{}{}



%% Flowchart stuff
%\tikzstyle{startstop} = [rectangle, rounded corners, minimum width=3cm, minimum height=1cm,text centered, draw=black]
%\tikzstyle{question} = [rectangle, minimum width=3cm, minimum height=1cm, text centered, draw=black]
%\tikzstyle{decision} = [trapezium, trapezium left angle=70, trapezium right angle=110, minimum width=3cm, minimum height=1cm, text centered, draw=black]
%\tikzstyle{question} = [rectangle, rounded corners, minimum width=3cm, minimum height=1cm,text centered, draw=black]
%\tikzstyle{process} = [rectangle, minimum width=3cm, minimum height=1cm, text centered, draw=black]
%\tikzstyle{decision} = [trapezium, trapezium left angle=70, trapezium right angle=110, minimum width=3cm, minimum height=1cm, text centered, draw=black]


\title[Dig-In:]{Mirrored surfaces}

\begin{document}
\begin{abstract}
  We think about how to reflect light of off surfaces. 
\end{abstract}
\maketitle

Suppose you have a function $F:\R^2\to\R$ via
\[
F(x,y) = x^2 + y^2
\]
In this case, the surface is called a \dfn{paraboloid}.  and we would
like to understand how light would reflect off such an object, if we
were to build it as a physical object.  Suppose $\vec{\ell}(t)$ is an
arbitrary (parametrized!) straight line, given by
\[
\vec{\ell}(t) = \vec{\ell}_0 + t \vec{v}
\]
and suppose the image of $\vec{\ell}(t)$ intersects the image of
$\vec{F}$ at a point $(a,b,c)$.

\begin{sageCell}
# The purple surface
f(x,y) = x^2 + y^2

# The red light beam
t = var('t')
ell0 = vector([0.85,0.5,1.5])
v = vector([-0.2,-0.1,-1])

plot3d( f, (x,-1,1), (y,-1,1) ) + parametric_plot3d( ell0 + t * v, (t,-0.5,2), color="red", thickness=0.1)
\end{sageCell}


When we zoom in close enough to the point $(a,b,c)$ where the light
ray hits the surface, we imagine that the mirrored surface is a plane.

\begin{question}
  What is the tangent plane to $F$ at the point $(a,b,c)$?
  \begin{prompt}
    \[
    z = \answer{2a(x-a)+2b(y-b)}
    \]
  \end{prompt}
\end{question}
When the light ray hits the surface, it is heading in the direction
\begin{multipleChoice}
  \choice{$\pp[x]{\vec{F}}$}
  \choice{$\pp[y]{\vec{F}}$}
  \choice[correct]{$\vec{v}$}
\end{multipleChoice}
This means that the reflected light will be heading in the direction
%% \begin{multipleChoice}
%%   \choice{$\vec{v} + 2 \proj_{\vec{v}}(\vec{n})}
%%   \choice{$\vec{v} - 2 \proj_{\vec{v}}(\vec{n})}
%%   \choice{$\vec{v} + 2 \proj_{\pp{\vec{F}}{x} \cross \pp{\vec{F}}{y}}(\vec{v})$}
%%   \choice[correct]{$\vec{v} - 2 \proj_{\pp{x}{\vec{F}} \cross \pp{y}{\vec{F}}}( \vec{v})$}
%% \end{multipleChoice}
Such formulas justify the time we spent thinking deeply about $\proj$.  And now we can plot all this together.
\begin{sageCell}
# The purple surface
a = var('a'); b = var('b')
f = vector([a,b,a^2 + b^2])

# The red light beam
t = var('t') ; ell0 = vector([0.85,0.5,1.5]) ; v = vector([0.1,-0.2,-1])

# The intersection of the surface and the light beam
solutions = solve( [f[i] == (ell0 + t*v)[i] for i in range(3)], [a,b,t], solution_dict=True)
solution = [s for s in solutions if t.subs(s) > 0][0]
xyz = f.subs(solution)

# The reflected vector
proj = lambda v, w: (v.dot_product(w))/(w.dot_product(w)) * w
n = derivative(f,a).cross_product( derivative(f,b) )
n = n.subs(solution)
rv = v - 2 * proj(v,n)

# Plot the surface, the incoming beam, and the outgoing beam
parametric_plot3d( f, (a,-1,1), (b,-1,1) ) + parametric_plot3d( ell0 + t * v, (t,-0.5,t.subs(solution)), color="red", thickness=0.1) + parametric_plot3d( xyz + t * rv, (t,0,2), color="red", thickness=0.1)
\end{sageCell}



\begin{example}[Vertical light]
  By playing around with the vector $\vec{v}$, i.e., the direction of
  the incoming light, we can discover some significant, qualitative
  features of reflections in paraboloids.

  Let $\vec{F}(x,y) = (x,y,x^2 + y^2)$, so that
  \begin{align*}
    \pp{\vec{F}}{x} &= \vector{ 1, 0, \answer{2x}} \\
    \pp{\vec{F}}{y} &= \vector{0, 1, \answer{2y}} \\
    \vec{n} &= \pp{\vec{F}}{x} \cross \pp{\vec{F}}{y} &= \vector{-2x, -2y, \answer{1}}. \\
  \end{align*}

  We want to focus in on the case of ``vertical light'' which arrives directly from above.  One way to 
  restrict to this situation would be to assume that both $\vec{v} \dotp \langle 1, 0, 0 \rangle = 0$ and
  $\vec{v} \dotp \vector{ 0, 1, 0 } = 0$, but to simplify the situation further, simply suppose
  that $\vec{v} = \vector{ 0, 0, -1 }$.  Then
  \begin{align*}
    \vec{v} - 2 \proj_{\vec{n}} (\vec{v}) 
    &= \vec{v} - 2 \frac{\vec{v} \dotp \vec{n}}{\vec{n} \dotp \vec{n}} \vec{n} \\
    &= \vec{v} - 2 \frac{\answer{-1}}{\vec{n} \dotp \vec{n}} \vec{n} \\
  \end{align*}
  This reflected vector simplies further to
  \[
    \frac{-4}{1 + 4x^2 + 4y^2} \vector{\answer{x}, \answer{y}, \answer{x^2 + y^2 - 1/4}}.
  \]
  The significance of this is that, starting at the point
  $(x,y,x^2 + y^2)$ and moving in the direction of this vector, we pass through the point 
  \[
    \left( 0, 0, \answer[given]{1/4} \right),
  \]
  meaning that \textit{every} vertical beam of light reflects off the
  paraboloid and is focused on common point!  This fact is what makes
  solar ovens and satellite dishes possible.
  
\end{example}

\end{document}
