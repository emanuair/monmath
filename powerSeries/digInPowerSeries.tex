\documentclass{ximera}

%\usepackage{todonotes}

\newcommand{\todo}{}

\usepackage{esint} % for \oiint
\ifxake%%https://math.meta.stackexchange.com/questions/9973/how-do-you-render-a-closed-surface-double-integral
\renewcommand{\oiint}{{\large\bigcirc}\kern-1.56em\iint}
\fi


\graphicspath{
  {./}
  {ximeraTutorial/}
  {basicPhilosophy/}
  {functionsOfSeveralVariables/}
  {normalVectors/}
  {lagrangeMultipliers/}
  {vectorFields/}
  {greensTheorem/}
  {shapeOfThingsToCome/}
  {dotProducts/}
  {partialDerivativesAndTheGradientVector/}
  {../productAndQuotientRules/exercises/}
  {../normalVectors/exercisesParametricPlots/}
  {../continuityOfFunctionsOfSeveralVariables/exercises/}
  {../partialDerivativesAndTheGradientVector/exercises/}
  {../directionalDerivativeAndChainRule/exercises/}
  {../commonCoordinates/exercisesCylindricalCoordinates/}
  {../commonCoordinates/exercisesSphericalCoordinates/}
  {../greensTheorem/exercisesCurlAndLineIntegrals/}
  {../greensTheorem/exercisesDivergenceAndLineIntegrals/}
  {../shapeOfThingsToCome/exercisesDivergenceTheorem/}
  {../greensTheorem/}
  {../shapeOfThingsToCome/}
  {../separableDifferentialEquations/exercises/}
}

\newcommand{\mooculus}{\textsf{\textbf{MOOC}\textnormal{\textsf{ULUS}}}}

\usepackage{tkz-euclide}\usepackage{tikz}
\usepackage{tikz-cd}
\usetikzlibrary{arrows}
\tikzset{>=stealth,commutative diagrams/.cd,
  arrow style=tikz,diagrams={>=stealth}} %% cool arrow head
\tikzset{shorten <>/.style={ shorten >=#1, shorten <=#1 } } %% allows shorter vectors

\usetikzlibrary{backgrounds} %% for boxes around graphs
\usetikzlibrary{shapes,positioning}  %% Clouds and stars
\usetikzlibrary{matrix} %% for matrix
\usepgfplotslibrary{polar} %% for polar plots
%\usepgfplotslibrary{fillbetween} %% to shade area between curves in TikZ
\usetkzobj{all}
\usepackage[makeroom]{cancel} %% for strike outs
%\usepackage{mathtools} %% for pretty underbrace % Breaks Ximera
%\usepackage{multicol}
\usepackage{pgffor} %% required for integral for loops



%% http://tex.stackexchange.com/questions/66490/drawing-a-tikz-arc-specifying-the-center
%% Draws beach ball
\tikzset{pics/carc/.style args={#1:#2:#3}{code={\draw[pic actions] (#1:#3) arc(#1:#2:#3);}}}



\usepackage{array}
\setlength{\extrarowheight}{+.1cm}
\newdimen\digitwidth
\settowidth\digitwidth{9}
\def\divrule#1#2{
\noalign{\moveright#1\digitwidth
\vbox{\hrule width#2\digitwidth}}}





\newcommand{\RR}{\mathbb R}
\newcommand{\R}{\mathbb R}
\newcommand{\N}{\mathbb N}
\newcommand{\Z}{\mathbb Z}

\newcommand{\sagemath}{\textsf{SageMath}}


%\renewcommand{\d}{\,d\!}
\renewcommand{\d}{\mathop{}\!d}
\newcommand{\dd}[2][]{\frac{\d #1}{\d #2}}
\newcommand{\pp}[2][]{\frac{\partial #1}{\partial #2}}
\renewcommand{\l}{\ell}
\newcommand{\ddx}{\frac{d}{\d x}}

\newcommand{\zeroOverZero}{\ensuremath{\boldsymbol{\tfrac{0}{0}}}}
\newcommand{\inftyOverInfty}{\ensuremath{\boldsymbol{\tfrac{\infty}{\infty}}}}
\newcommand{\zeroOverInfty}{\ensuremath{\boldsymbol{\tfrac{0}{\infty}}}}
\newcommand{\zeroTimesInfty}{\ensuremath{\small\boldsymbol{0\cdot \infty}}}
\newcommand{\inftyMinusInfty}{\ensuremath{\small\boldsymbol{\infty - \infty}}}
\newcommand{\oneToInfty}{\ensuremath{\boldsymbol{1^\infty}}}
\newcommand{\zeroToZero}{\ensuremath{\boldsymbol{0^0}}}
\newcommand{\inftyToZero}{\ensuremath{\boldsymbol{\infty^0}}}



\newcommand{\numOverZero}{\ensuremath{\boldsymbol{\tfrac{\#}{0}}}}
\newcommand{\dfn}{\textbf}
%\newcommand{\unit}{\,\mathrm}
\newcommand{\unit}{\mathop{}\!\mathrm}
\newcommand{\eval}[1]{\bigg[ #1 \bigg]}
\newcommand{\seq}[1]{\left( #1 \right)}
\renewcommand{\epsilon}{\varepsilon}
\renewcommand{\phi}{\varphi}


\renewcommand{\iff}{\Leftrightarrow}

\DeclareMathOperator{\arccot}{arccot}
\DeclareMathOperator{\arcsec}{arcsec}
\DeclareMathOperator{\arccsc}{arccsc}
\DeclareMathOperator{\si}{Si}
\DeclareMathOperator{\scal}{scal}
\DeclareMathOperator{\sign}{sign}


%% \newcommand{\tightoverset}[2]{% for arrow vec
%%   \mathop{#2}\limits^{\vbox to -.5ex{\kern-0.75ex\hbox{$#1$}\vss}}}
\newcommand{\arrowvec}[1]{{\overset{\rightharpoonup}{#1}}}
%\renewcommand{\vec}[1]{\arrowvec{\mathbf{#1}}}
\renewcommand{\vec}[1]{{\overset{\boldsymbol{\rightharpoonup}}{\mathbf{#1}}}}
\DeclareMathOperator{\proj}{\mathbf{proj}}
\newcommand{\veci}{{\boldsymbol{\hat{\imath}}}}
\newcommand{\vecj}{{\boldsymbol{\hat{\jmath}}}}
\newcommand{\veck}{{\boldsymbol{\hat{k}}}}
\newcommand{\vecl}{\vec{\boldsymbol{\l}}}
\newcommand{\uvec}[1]{\mathbf{\hat{#1}}}
\newcommand{\utan}{\mathbf{\hat{t}}}
\newcommand{\unormal}{\mathbf{\hat{n}}}
\newcommand{\ubinormal}{\mathbf{\hat{b}}}

\newcommand{\dotp}{\bullet}
\newcommand{\cross}{\boldsymbol\times}
\newcommand{\grad}{\boldsymbol\nabla}
\newcommand{\divergence}{\grad\dotp}
\newcommand{\curl}{\grad\cross}
%\DeclareMathOperator{\divergence}{divergence}
%\DeclareMathOperator{\curl}[1]{\grad\cross #1}
\newcommand{\lto}{\mathop{\longrightarrow\,}\limits}

\renewcommand{\bar}{\overline}

\colorlet{textColor}{black}
\colorlet{background}{white}
\colorlet{penColor}{blue!50!black} % Color of a curve in a plot
\colorlet{penColor2}{red!50!black}% Color of a curve in a plot
\colorlet{penColor3}{red!50!blue} % Color of a curve in a plot
\colorlet{penColor4}{green!50!black} % Color of a curve in a plot
\colorlet{penColor5}{orange!80!black} % Color of a curve in a plot
\colorlet{penColor6}{yellow!70!black} % Color of a curve in a plot
\colorlet{fill1}{penColor!20} % Color of fill in a plot
\colorlet{fill2}{penColor2!20} % Color of fill in a plot
\colorlet{fillp}{fill1} % Color of positive area
\colorlet{filln}{penColor2!20} % Color of negative area
\colorlet{fill3}{penColor3!20} % Fill
\colorlet{fill4}{penColor4!20} % Fill
\colorlet{fill5}{penColor5!20} % Fill
\colorlet{gridColor}{gray!50} % Color of grid in a plot

\newcommand{\surfaceColor}{violet}
\newcommand{\surfaceColorTwo}{redyellow}
\newcommand{\sliceColor}{greenyellow}




\pgfmathdeclarefunction{gauss}{2}{% gives gaussian
  \pgfmathparse{1/(#2*sqrt(2*pi))*exp(-((x-#1)^2)/(2*#2^2))}%
}


%%%%%%%%%%%%%
%% Vectors
%%%%%%%%%%%%%

%% Simple horiz vectors
\renewcommand{\vector}[1]{\left\langle #1\right\rangle}


%% %% Complex Horiz Vectors with angle brackets
%% \makeatletter
%% \renewcommand{\vector}[2][ , ]{\left\langle%
%%   \def\nextitem{\def\nextitem{#1}}%
%%   \@for \el:=#2\do{\nextitem\el}\right\rangle%
%% }
%% \makeatother

%% %% Vertical Vectors
%% \def\vector#1{\begin{bmatrix}\vecListA#1,,\end{bmatrix}}
%% \def\vecListA#1,{\if,#1,\else #1\cr \expandafter \vecListA \fi}

%%%%%%%%%%%%%
%% End of vectors
%%%%%%%%%%%%%

%\newcommand{\fullwidth}{}
%\newcommand{\normalwidth}{}



%% makes a snazzy t-chart for evaluating functions
%\newenvironment{tchart}{\rowcolors{2}{}{background!90!textColor}\array}{\endarray}

%%This is to help with formatting on future title pages.
\newenvironment{sectionOutcomes}{}{}



%% Flowchart stuff
%\tikzstyle{startstop} = [rectangle, rounded corners, minimum width=3cm, minimum height=1cm,text centered, draw=black]
%\tikzstyle{question} = [rectangle, minimum width=3cm, minimum height=1cm, text centered, draw=black]
%\tikzstyle{decision} = [trapezium, trapezium left angle=70, trapezium right angle=110, minimum width=3cm, minimum height=1cm, text centered, draw=black]
%\tikzstyle{question} = [rectangle, rounded corners, minimum width=3cm, minimum height=1cm,text centered, draw=black]
%\tikzstyle{process} = [rectangle, minimum width=3cm, minimum height=1cm, text centered, draw=black]
%\tikzstyle{decision} = [trapezium, trapezium left angle=70, trapezium right angle=110, minimum width=3cm, minimum height=1cm, text centered, draw=black]


\outcome{Give the definition of a power series.}
\outcome{Find the interval and radius of convergence of a power series.}
\outcome{Express functions as power series.}
\outcome{Express power series as closed-form functions.}
\outcome{Differentiate and integrate power series.}

\title[Dig-In:]{Power series}

\begin{document}
\begin{abstract}
  Infinite series can represent functions.
\end{abstract}
\maketitle


\begin{image}
\begin{tikzpicture}

\begin{axis}
	[
	domain=-6:3,
	axis x line = middle,
	axis line style=-,
	axis y line = none,
	xmin=-6.5,
	xmax=3.5,
	ymin=-1,
	ymax=1,
	xtick={-1.5},
	xticklabels={},
	]
	
	\draw[very thick,penColor] (-3,0) -- (0,0);
	\draw[->,very thick,penColor2] (-4.5,0) -- (-6.5,0);
	\draw[->,very thick,penColor2] (1.5,0) -- (3.5,0);
	\draw[very thick,penColor3] (-4.5,0) -- (-3,0);
	\draw[very thick,penColor3] (0,0) -- (1.5,0);
	
	\node at (-4.5,0) [scale=.5,shape=circle, fill=penColor2] {};
	\node at (-4.5,-.05) [below, penColor2] {$-4.5$};
	\node at (-3,0) [scale=.5,shape=circle, fill=white, draw=penColor] {};
	\node at (-3,-.05) [below, penColor] {$-3$};
	%\node at (-1.5,0) [scale=.5,shape=circle, fill=penColor] {};
	\node at (-1.5,-.05) [below, penColor] {$-1.5$};
	\node at (0,0) [scale=.5,shape=circle, fill=white, draw=penColor] {};
	\node at (0,-.05) [below, penColor] {$0$};
	\addplot[draw=none] coordinates {(-4.5,0)};
	\addplot[draw=none] coordinates {(1.5,0)};
%	\node[excl] at (1.5,0) [scale=.5,shape=circle, fill=penColor2] {};
	\node at (1.5,-.05) [below, penColor2] {$1.5$};
	\draw [penColor2,thick,decoration={brace,mirror,raise=2em},decorate] 
        (axis cs:-6.5,0) --
        (axis cs:-4.5,0);
    \draw [penColor,thick,decoration={brace,mirror,raise=2em},decorate] 
        (axis cs:-3,0) --
        (axis cs:0,0);
    \draw [penColor2,thick,decoration={brace,mirror,raise=2em},decorate] 
        (axis cs:1.5,0) --
        (axis cs:3.5,0);
    \node[penColor2] at (-5.5,-.4) {diverges};
    \node[penColor] at (-1.5,-.41) {converges};
    \node[penColor2] at (2.5,-.4) {diverges};
    \draw [penColor3,thick,decoration={brace,raise=1.5em},decorate] 
        (axis cs:0,0) --
        (axis cs:1.5,0);
    \draw [penColor3,thick,decoration={brace,raise=1.5em},decorate] 
        (axis cs:-4.5,0) --
        (axis cs:-3,0);
    \node[penColor3] at (-1.5,.3) {may or may not converge};
\end{axis}

\end{tikzpicture}
\end{image}



If we refuse to truncate a Taylor polynomial, and instead allow it to
be a series (an infinite sum) we call it a \textit{power series}. 

\begin{definition}
  A \dfn{power series} is a series of the form
  \[
  \sum_{k=0}^\infty a_k(x-c)^k
  \]
  where the $a_k$'s are the \dfn{coefficients} and $c$ is the
  \dfn{center}.
\end{definition}

\begin{question}
  Which of the following are power series functions?
  \begin{selectAll}
    \choice[correct]{$f(x) = 0$}
    \choice[correct]{$f(x) = -9$}
    \choice[correct]{$f(x) = 3x+1$}
    \choice{$f(x) = x^{1/2}-x +8$}
    \choice{$f(x) = -4x^{-3}+5x^{-1}+7-18x^2$}
    \choice{$f(x) = x^{-3}+x^{-2}+x^{-1}+1+x+x^2+x^3 +\cdots$}
    \choice{$f(x) = \frac{x^2 - 3x + 2}{x-2}$}
    \choice[correct]{$f(x) = x^7-32x^6-\pi x^3+45/84$}
    \choice[correct]{$f(x) = x^{10} + x^{20} + x^{30} + \cdots$}    
  \end{selectAll}
  \begin{hint}
    Every polynomial is a power series.
  \end{hint}
\end{question}

Here are four basic power series (centered at zero) that every
mathematician knows.  
\begin{align*}
           e^x &= 1 + x + \frac{x^2}{2!} + \frac{x^3}{3!} + \cdots &|x|< \infty\\
       \sin(x) &= x - \frac{x^3}{3!} + \frac{x^5}{5!} -\frac{x^7}{7!} + \cdots &|x|< \infty\\
       \cos(x) &= 1-\frac{x^2}{2!} + \frac{x^4}{4!} -\frac{x^6}{6!} + \cdots &|x|< \infty\\
 \frac{1}{1-x} &= 1+ x+ x^2 + x^3 + \cdots &|x|< 1
\end{align*}
Next to each of the series, we list an interval which will correspond
to the domain of the series.  Using power series we can ``read-off''
properties of functions. Here are some examples.
\begin{itemize}
\item We can easily see that $e^0 =1$, $\sin(0)=0$, and $\cos(0) =1$.
\item Since every power of $x$ in the power series for sine is odd, we
  can see that sine is an odd function. Likewise, since every power of
  $x$ in the power series for cosine is even, we can see cosine is an
  even function.
\item Limits like
  \[
  \lim_{x\to 0}\frac{\sin(x)}{x} = 1\qquad\text{and}\qquad \lim_{x\to 0} \frac{\cos(x)-1}{x} = 0
  \]
  are ``easy'' to compute, since they can be rewritten as follows.
  \begin{align*}
    \lim_{x\to 0}\frac{\sin(x)}{x} &=\lim_{x\to 0} \frac{x - \frac{x^3}{3!} + \frac{x^5}{5!} -\frac{x^7}{7!} + \cdots}{x}\\
    &=\lim_{x\to 0} \left(1 - \frac{x^2}{3!} + \frac{x^4}{5!} -\frac{x^6}{7!} + \cdots\right)\\
    &=1,
  \end{align*}
  and
  \begin{align*}
    \lim_{x\to 0} \frac{\cos(x)-1}{x}&=\lim_{x\to 0} \frac{1-\frac{x^2}{2!} + \frac{x^4}{4!} -\frac{x^6}{6!} + \cdots-1}{x}\\
    &=\lim_{x\to 0} \left(-\frac{x}{2!} + \frac{x^3}{4!} -\frac{x^5}{6!} + \cdots\right)\\
    &=0.
  \end{align*}
\item Power series give us methods to \textit{actually compute} values
  for these functions.
\end{itemize}

\section{Convergence of power series}

You may have noticed a small caveat above.
\[
\frac{1}{1-x} = 1+ x+ x^2 + x^3 + \cdots \qquad |x|< 1
\]
The caveat is the ``$|x|<1$,''  which we referred to as something like the 
domain of the function. This restriction is required because if
our formula is true, then for any number $r$, provided that $|r|<1$, we have
\[
\frac{1}{1-r} = 1+ r+ r^2 + r^3 + \cdots,
\]
and the expression on the right-hand side of the equation above is a
\index{geometric series}geometric series! As we've learned, geometric
series only converge when the common ratio (in this case $r$) is
between $-1$ and $1$ noninclusive. If we look at a graph of $y = \frac{1}{1-x}$ along with a graph of $y = 1+ x+ x^2 + x^3 + \cdots$ we see
\begin{image}
    \begin{tikzpicture}
    \begin{axis}[
        xmin=-3,xmax=3,ymin=-3,ymax=3,
        axis lines=center,
        width=6in,
        height=3in,
        xlabel=$x$, ylabel=$y$,
        every axis y label/.style={at=(current axis.above origin),anchor=south},
        every axis x label/.style={at=(current axis.right of origin),anchor=west},
      ]        
      \addplot [very thick, penColor, samples=100,smooth, domain=(-3:.8)] {1/(1-x)};
      \addplot [very thick, penColor, samples=100,smooth, domain=(1.2:3)] {1/(1-x)};
      \addplot [very thick, penColor2, samples=100,smooth, domain=(-1:.8)] {1/(1-x)};
      \addplot[color=penColor2,fill=white,only marks,mark=*] coordinates{(-1,1/2)};  %% open hole
      \node at (axis cs:1.5,2) {$1+ x+ x^2 + x^3 + \cdots$};
    \end{axis}
  \end{tikzpicture}
\end{image}

\begin{question}
  True or False:
  \[
  \frac{4}{3} = \frac{1}{1-(1/4)} = 1 + \left(\frac{1}{4}\right)+ \left(\frac{1}{4}\right)^2+ \left(\frac{1}{4}\right)^3 + \cdots
  \]
  \begin{prompt}
  \begin{multipleChoice}
    \choice[correct]{true}
    \choice{false}
  \end{multipleChoice}
  \end{prompt}
  \begin{question}
    True or False:
    \[
    \frac{1}{-3} = \frac{1}{1-4} = 1 + 4+ 4^2+ 4^3 + \cdots
    \]
    \begin{prompt}
      \begin{multipleChoice}
        \choice{true}
        \choice[correct]{false}
      \end{multipleChoice}
    \end{prompt}
\end{question}
\end{question}

Our next theorem tells us what possible scenarios we could encounter
when investigating convergence of power series.

\begin{theorem}[Convergence of Power Series]\index{convergence of power series}\index{power series!convergence}
  Consider the power series
  \[
  \sum_{n=0}^\infty a_n(x-c)^n.
  \]
  Exactly one of the following is true:
\begin{enumerate}
\item The series converges only at $x=c$.
\item There is an $R>0$ such that the series converges for all $x$ in	
  $(c-R,c+R)$ and diverges for all $x<c-R$ and $x>c+R$.
\item The series converges for all $x$.
\end{enumerate}
\end{theorem}

\begin{question}
  True or False: A power series
  \[
  \sum_{k=0}^\infty a_k(x-c)^k
  \]
  \textbf{always} converges when $x=c$.
  \begin{prompt}
    \begin{multipleChoice}
      \choice[correct]{true}
      \choice{false}
    \end{multipleChoice}
  \end{prompt}
  \begin{feedback}
    If $x=c$, then
    \begin{align*}
      \sum_{k=0}^\infty a_k(x-c)^k &= \sum_{k=0}^\infty a_k(c-c)^k \\
      &= \sum_{k=0}^\infty a_k(0)^k.
    \end{align*}
  \end{feedback}
  \begin{question}
    True or False: If 
    \[
    f(x) = \sum_{k=0}^\infty a_k(x-c)^k,
    \]
    then $f(c) = 0$. 
  \begin{prompt}
    \begin{multipleChoice}
      \choice{true}
      \choice[correct]{false}
    \end{multipleChoice}
  \end{prompt}
  \begin{feedback}
    \begin{align*}
      f(c) &= \sum_{k=0}^\infty a_k(c-c)^k \\
      &= \sum_{k=0}^\infty a_k(0)^k\\
      &= a_0
    \end{align*}
    since $0^k = 0$ when $k\ne 0$ and $0^0 = 1$.
  \end{feedback} 
\end{question}
\end{question}

Power series are both similar to and different from the series we've 
previously studied. When we fix some value for $x$, we are working 
with the sort of series we've already 
studied - a series of numbers.  In this way, we can use all of our previous 
tools for working with series.  We can also let $x$ be a variable, and consider 
our power series as a function.
 Because power series can define functions, we no longer exclusively
talk about convergence at a point, instead we talk about the
\textit{radius} and \textit{interval} of convergence.

\begin{definition}
  \hfil
  \begin{itemize}
    \item If a power series converges absolutely for all $x$, then its
      \dfn{radius of convergence} is said to be $\infty$ and the
      \dfn{interval of convergence} is $(-\infty,\infty)$.
    \item If a power series converges absolutely for all $x$ in
      $(c-R,c+R)$ and diverges for all $x<c-R$ and $x>c+R$, then its
      \dfn{radius of convergence} is said to be $R$ and the
      \dfn{interval of convergence} is one of the following:
      \[
      (c-R,c+R),\quad [c-R,c+R),\quad (c-R,c+R],\quad [c-R,c+R].
      \]
      \item If a power series converges only at one value $x=a$, then 
      its \dfn{radius of convergence} is said to be $0$ and the series does not have an 
      \dfn{interval of convergence}.
  \end{itemize}
\end{definition}

In the previous definition, the interval of convergence depends on the series.  We must 
separately consider the behavior of a power series at the endpoints of its interval 
of convergence.  In other words, we plug in values for $x$, and consider the series 
as a series of numbers!

\begin{question}
  Suppose you know that
  \[
  \sum_{n=0}^\infty a_n (x-3)^n
  \]
  converges when $x =7$ and diverges when $x = -1$. Must the series
  converge at $x=4$?
  \begin{prompt}
    \begin{multipleChoice}
      \choice[correct]{yes}
      \choice{no}
      \choice{there is not enough information}
    \end{multipleChoice}
  \end{prompt}
  \begin{feedback}
    Since we know that every power series converges either exactly at
    a single point or on an interval, we see that this power series
    \textbf{must} converge with radius of convergence $R=4$.
  \end{feedback}
\end{question}

How do we check for radius of convergence? Two old friends can come to
the rescue: the ratio and root tests.

\begin{example}
  Consider the power series:
  \[
  \sum_{n=0}^\infty \frac{x^n}{n!} = 1 + x + \frac{x^2}{2!} + \frac{x^3}{3!} + \cdots\\
  \]
  Determine the radius and interval of convergence.
  \begin{explanation}
    For this power series we will use the ratio test. Since the ratio test 
    requires positive terms, we must look at the absolute values of the terms in the
    series.
    \begin{align*}
      \lim_{n\to\infty} \frac{\frac{|x|^{n+1}}{(n+1)!}}{\frac{|x|^n}{n!}}
      &= \lim_{n\to\infty} \frac{|x|^{n+1}n!}{(n+1)!|x|^n}\\
      &= \lim_{n\to\infty} \frac{|x|}{n}.
    \end{align*}
    Now, for any \textbf{fixed} value of $x$, we have that
    \[
    \lim_{n\to\infty} \frac{|x|}{n} = \answer[given]{0},
    \]
    since we recall that $|x|$ is a constant in this limit and its value 
    does not affect the value of the limit.  Hence, the radius of convergence for $\sum_{n=0}^\infty
    \frac{x^n}{n!}$ is $R=\infty$, and the interval of convergence is $(-\infty, \infty)$.
  \end{explanation}
\end{example}

While the ratio and root test are good for determining the radius of
convergence of a power series, they are useless for determining
convergence at the end-points of the interval. Let's see an example:

\begin{example}
  Consider the power series:
  \[
  \sum_{n=1}^\infty \frac{(x-1)^n}{n \cdot 9^n} = \frac{(x-1)}{9} + \frac{(x-1)^2}{2\cdot 9^2} + \frac{(x-1)^3}{3\cdot 9^3} + \cdots\\
  \]
  Determine the radius and interval of convergence.
  \begin{explanation}
    Here, let's start with the root test. Again, we must first use the absolute value of the terms in the series:
    \begin{align*}
      \lim_{n\to\infty}\sqrt[n]{\frac{|x-1|^n}{n \cdot 9^n}} &= \lim_{n\to\infty}\sqrt[n]{\frac{|x-1|^n}{n \cdot 9^n}}\\
      &= \lim_{n\to\infty} \frac{|x-1|}{\sqrt[n]{n} \cdot 9}\\
      &= \frac{|x-1|}{\sqrt[n]{n} \cdot 9} \lim_{n\to\infty} n^{-1/n}.
    \end{align*}
    Using logarithms and L'H\^opital's rule, we can show that
    \[
    \lim_{n\to\infty} n^{-1/n} = \answer[given]{1}.
    \]
    Hence
    \[
    \lim_{n\to\infty}\sqrt[n]{\frac{|x-1|^n}{n \cdot 9^n}} = \frac{|x-1|}{9}.
    \]
    The root test gives us convergence when this limit is between $-1$ and $1$. 
    In other words, the series converges absolutely when
    \begin{align*}
    \frac{|x-1|}{9} &<\answer[given]{1}\\
    |x-1| &< \answer[given]{9}.
    \end{align*}
    However,
    \[
    |x-1| < 9 \qquad\text{means that}\qquad \answer[given]{-9} < x-1 < \answer[given]{9}
    \]
    and so adding $1$ to all sides of the inequality, we need $x$ such that
    \[
    \answer[given]{-8} < x < \answer[given]{10}.
    \]
    Since our power series is centered at $x=1$, the radius of
    convergence is $R=9$. However, the root test (and ratio test) is inconclusive 
    at the end points $x=-8$ and $x=10$. For this, we need to investigate separately
    the following \textit{two} series, found by plugging in $x = -8$ and $x=10$.
    \[
    \sum_{n=1}^\infty \frac{(-8-1)^n}{n \cdot 9^n}\qquad\text{and}\qquad \sum_{n=1}^\infty \frac{(10-1)^n}{n \cdot 9^n}
    \]
    For the first, where $x=-8$, note that
    \begin{align*}
      \sum_{n=1}^\infty \frac{(-8-1)^n}{n \cdot 9^n} &= \sum_{n=1}^\infty \frac{(-9)^n}{n \cdot 9^n}\\
      &= \sum_{n=1}^\infty \frac{(-1)^n}{n}.
    \end{align*}
    This is the alternating harmonic series, which we know converges. So our power series converges at $x= -8$.
    For the second, where $x=10$, note that
    \begin{align*}
      \sum_{n=1}^\infty \frac{(10-1)^n}{n \cdot 9^n} &= \sum_{n=1}^\infty \frac{(9)^n}{n \cdot 9^n}\\
      &= \sum_{n=1}^\infty \frac{1}{n}.
    \end{align*}
    This is the harmonic series, which we know diverges. So our
    power series diverges at $x= 10$. Hence the interval of
    convergence for $\sum_{n=1}^\infty \frac{(x-1)^n}{n \cdot 9^n}$ must include everything 
    between $-8$ and $10$, as well as $-8$, but does not include $10$.  In other words, the 
    interval of convergence is $[-8,10)$.
  \end{explanation}
\end{example}

Let's work through an example of a power series that only converges at a single
point.

\begin{example}
  Consider the power series:
  \[
  \sum_{n=0}^\infty n!(x+7)^n = 1 + (x+7) + 2(x+7)^2 + 6(x+7)^3 + \cdots.
  \]
  Determine the radius and interval of convergence.
  \begin{explanation}
    Here we'll use the ratio test, looking at the absolute value of
    the terms in the series.
    \[
    \lim_{n\to\infty} \frac{(n+1)!|x+7|^{n+1}}{n!|x+7|^n}= \lim_{n\to\infty} \answer[given]{(n+1)}|x+7|
    \]
    This limit diverges unless $x=-7$, the center of the power
    series. The the radius of convergence is $R=0$, and there is no
    interval of convergence, since the series only converges at a
    single point.
  \end{explanation}
\end{example}



\section{New power series from old}

With the basic power series above, we can produce new power series via
algebraic manipulation.

\begin{theorem}[Algebra of Power Series]\index{power series!algebra of}
  Let
  \begin{align*}
    f(x) &= \sum_{n=0}^\infty a_nx^n\\
    g(x) &= \sum_{n=0}^\infty b_nx^n
  \end{align*}
  converge absolutely for $|x|<R$, and let $h(x)$ be a continuous function.
  \begin{itemize}
	\item $f(x)\pm g(x) = \sum_{n=0}^\infty (a_n\pm b_n)x^n$ \quad for $|x|<R$.
	%\item $f(x)g(x) = \left(\sum_{n=0}^\infty a_nx^n\right)\left(\sum_{n=0}^\infty b_nx^n\right) = \sum_{n=0}^\infty\big(a_0b_n+a_1b_{n-1}+\ldots a_nb_0\big)x^n$ for $|x|<R$.
	\item $\begin{aligned}[t]
	f(x)g(x) &= \left(\sum_{n=0}^\infty a_nx^n\right)\left(\sum_{n=0}^\infty b_nx^n\right)\\
	      &= \sum_{n=0}^\infty\big(a_0b_n+a_1b_{n-1}+\dots + a_nb_0\big)x^n
		\end{aligned}$ for $|x|<R$.%\hfill
	\item $f\big(h(x)\big) = \sum_{n=0}^\infty a_n\big(h(x)\big)^n$ \quad for $|h(x)|<R$.
  \end{itemize}
\end{theorem}

In our first example we will derive Euler's famous formula
\[
e^{ix} = \cos(x) + i \sin(x),
\]
where $i$ is the number $i^2=-1$.

\begin{example}
  Use power series to show
  \[
  \cos(x) + i \sin(x) = e^{ix} 
  \]
  \begin{explanation}
    We start by writing the relevant power series for cosine
    \[
    \cos(x) = 1-\frac{x^2}{2!} + \frac{x^4}{4!} -\frac{x^6}{6!} + \cdots
    \]
    and now we consider
    \[
    i \sin(x) = ix - \frac{ix^3}{3!} + \frac{ix^5}{5!} -\frac{i x^7}{7!} + \cdots.
    \]
    Adding these power series (and ordering the terms by degree) we find
    \[
    1+ ix -\frac{x^2}{2!} - \frac{ix^3}{3!} + \frac{x^4}{4!} + \frac{ix^5}{5!} -\frac{x^6}{6!} -\frac{i x^7}{7!} + \cdots.
    \]
    Since $i^0=\answer[given]{1}$,
    $i^1= \answer[given]{i}$,
    $i^2 = \answer[given]{-1}$,
    $i^3 = \answer[given]{-i}$,
    $i^4 = \answer[given]{1}$, and so on with a repeating pattern we may now write
    \begin{align*}
      1+ ix &+\frac{(ix)^2}{2!} + \frac{(ix)^3}{3!} + \frac{(ix)^4}{4!} + \frac{(ix)^5}{5!} +\frac{(ix)^6}{6!} +\frac{(i x)^7}{7!} + \cdots\\
      &= e^{ix}.
    \end{align*}
  \end{explanation}
\end{example}

Euler's formula $e^{ix} = \cos(x) + i \sin(x)$ allows us to produce
(by setting $x=\pi$) the amazing identity:
\[
e^{i \pi } + 1 = 0.
\]
This identity combines the fundamental constants, $0$, $1$, $i$,
$\pi$ and $e$, along with the fundamental operations of addition,
multiplication, and exponentiation!

\begin{example}
  Use power series to give evidence (by looking at the first $4$ terms
  of a power series) for the double angle formula:
  \[
  2\sin(x) \cos(x) = \sin(2x)
  \]
  \begin{explanation}
    Write with me.
    \begin{align*}
      \sin(x) &= x - \frac{x^3}{3!} + \frac{x^5}{5!} -\frac{x^7}{7!} + \cdots &|x|< \infty\\
      \cos(x) &= 1-\frac{x^2}{2!} + \frac{x^4}{4!} -\frac{x^6}{6!} + \cdots &|x|< \infty
    \end{align*}
    Multiplying we find
    \begin{image}
      \begin{tikzpicture}
        \node at (0,0) {$
          \begin{aligned}
            \sin(x) &\cos(x) \\
            &= x + \left(\frac{-1}{3!} + \frac{-1}{2!}\right)x^3 + \left(\frac{1}{5!} +\frac{1}{3!\cdot 2!}+ \frac{1}{4!}\right)x^5
            +\left(\frac{-1}{7!} +\frac{-1}{5!\cdot 2!}+\frac{-1}{3!\cdot 4!}+ \frac{-1}{6!}\right)x^7 +\cdots\\
            &=x + \left(\frac{-1}{3!} + \frac{-3}{3!}\right)x^3 + \left(\frac{2!}{5!\cdot 2!} +\frac{5\cdot4}{5!\cdot 2!}+ \frac{5\cdot 2!}{5!\cdot 2!}\right)x^5
            +\left(\frac{-3!}{7!\cdot3!} +\frac{-7\cdot 6\cdot 3}{7!\cdot 3!}+\frac{-7\cdot6\cdot5}{7!\cdot 3!}+ \frac{-7\cdot 3!}{7!\cdot 3!}\right)x^7 +\cdots\\
            &= x + \left(\frac{-2^2}{3!}\right)x^3 + \left(\frac{2^4}{5!}\right)x^5+\left(\frac{-2^6}{7!}\right)x^7 +\cdots
          \end{aligned}
          $};
      \end{tikzpicture}
    \end{image}
    So $2\sin(x)\cos(x)$
    \begin{align*}
      &= 2x + \left(\answer[given]{\frac{-2^3}{3!}}\right)x^3 + \left(\answer[given]{\frac{2^5}{5!}}\right)x^5+\left(\frac{-2^7}{7!}\right)x^7 +\cdots\\
      &= (2x) -\frac{(2x)^3}{3!} + \frac{(2x)^5}{5!}-\frac{(2x)^7}{7!} +\cdots\\
      &= \sin(2x).
    \end{align*}
  \end{explanation}
\end{example}



\begin{example}
  Use the power series for $\frac{1}{1-x}$ to give a power series for $\frac{1}{1+x^2}$
  \begin{explanation}
    We know that
    \[
    \frac{1}{1-x}= 1+ x+ x^2 + x^3 + \cdots\qquad|x|< 1.
    \]
    Since $-x^2$ is continuous when $|x|<1$, we can compose this with $\frac{1}{1-x}$ to see
    \begin{align*}
    \frac{1}{1-(-x^2)} &=\frac{1}{1+x^2}\\
    &= \answer[given]{1 - x^2 + x^4} - x^6 + \cdots &|x|< 1\\
    &= \sum_{n=0}^\infty (-1)^n x^{2n} &|x|< 1
    \end{align*}
  \end{explanation}
\end{example}

\begin{theorem}[Derivatives and Indefinite Integrals of Power Series]\index{power series!derivatives and integrals}
  Let
  \[
  f(x) = \sum_{n=0}^\infty a_n(x-c)^n
  \]
  be a function defined by a power series, with radius of convergence $R$.
  \begin{itemize}
  \item $f(x)$ is continuous and differentiable on $(c-R,c+R)$.
  \item	$f'(x) = \sum_{n=1}^\infty a_n\cdot n\cdot (x-c)^{n-1}$, with radius of convergence $R$.
  \item	$\int f(x) \d x = C+\sum_{n=0}^\infty a_n\frac{(x-c)^{n+1}}{n+1}$, with radius of convergence $R$.
  \end{itemize}
\end{theorem}
%% https://gowers.wordpress.com/2014/02/22/differentiating-power-series/
A few notes about the theorem above:
\begin{itemize}
\item The theorem states that differentiation and integration do not
  change the radius of convergence. It does not state anything about
  the \textit{interval} of convergence. They are not always the same.  Check the endpoints!
\item Notice how the summation for $f'(x)$ starts with $n=1$. This is
  because the derivative of the constant term $a_0$ of $f(x)$ is $0$.
\item Differentiation and integration are simply calculated
  term-by-term using the power rule.
\end{itemize}

Let's see an example.

\begin{example}
  Use the power series for
  \[
  \frac{1}{1+x^2} = 1 - x^2 + x^4 - x^6 + \cdots \qquad|x|< 1 
  \]
  to find a power series for $\arctan(x)$. Given the radius and interval of convergence.
  \begin{explanation}
    Since
    \[
    \int\frac{1}{1+x^2} \d x = \arctan(x)+C
    \]
    we can find the desired power series by integrating. Write with me.
    \[
    \int \left(1 - x^2 + x^4 - x^6 + \cdots\right)\d x = \answer[given]{x - \frac{x^3}{3} + \frac{x^5}{5}} - \frac{x^7}{7} + \cdots +C
    \]
    Since $\arctan(0) = 0$, $C=0$, and we have our desired power
    series. Its radius of convergence is $R=1$. However,
    we recall that the interval of convergence may be different from the original series, so we 
    set out to check the endpoints. First note that our power series can be written in summation notation as
    \[
    \sum_{n=0}^\infty \frac{(-1)^n x^{2n+1}}{2n+1}.
    \]
    If $x=1$ or $x=-1$ we can see that this sequence is
    \[
    \sum_{n=0}^\infty \frac{(-1)^n}{2n+1}\qquad\text{or}\qquad\sum_{n=0}^\infty \frac{(-1)^{n+1}}{2n+1}
    \]
    In both cases, the series converges by the alternating series
    test. Hence the interval of convergence is $[\answer[given]{-1},\answer[given]{1}]$.
  \end{explanation}
\end{example}



\end{document}
